%_HANDOUT_\documentclass[handout]{beamer}
\documentclass{beamer}      %_SLIDES_ONLY_%
\usepackage[utf8]{inputenc}
\usepackage{lingmacros}
\usepackage{multirow}
\usepackage{qtree}
\usepackage{fancybox}
\usepackage{listings}
\usepackage{url}
\usepackage{xmpmulti}
\usepackage{hyperref}
\usepackage{totcount}
\usepackage{tikz}
\usepackage{calc}
\usepackage{amsfonts}
\usepackage{euler}
\usepackage[vlined,ruled,slide,linesnumbered]{algorithm2e}
\usepackage{cc-presentation}
\usetikzlibrary{calc}
%\usetikzlibrary{external}
%\tikzexternalize

%_HANDOUT_\usepackage{handoutWithNotes}
%_HANDOUT_\pgfpagesuselayout{4 on 1 with notes}[a4paper,border shrink=5mm]

\newcommand{\mywebpage}{http://www.let.rug.nl/coltekin}
\newcommand{\clcgwebpage}{http://www.rug.nl/let/onderzoek/onderzoekinstituten/clcg/index}

\author[\href{\mywebpage}{Ç. Çöltekin,~CLCG}]%
{\href{\mywebpage}{Çağrı Çöltekin}\\ \vskip2mm\tiny\tt
c.coltekin@rug.nl} \institute{\href{\clcgwebpage}{Center for Language
and Cognition\\ University of Groningen}}
\title[Segmentation]{Discovering Words in Continuous
Speech\\\vspace{2mm}
\small Computational Simulations of Speech Segmentation} 
\date{June 2011}

\begin{document}

\begin{frame}[plain]
\maketitle
\end{frame}

\mode<article>{
This talk is about computational methods of segmentation of continuous
speech into linguistically useful units, such as words. As competent
users of a language we take segmentation as granted. However, in
fluent speech, the word boundaries are not marked. We do not actually
have a break after every word. This is probably one of the first
difficulties of learning a foreign language. When you listen to a
foreign language, you generally do not have any clue where the words
start and end. And some languages are written without spaces, so
segmentation is a step necessary for any further language processing
tasks. In this talk we will concentrate on the first problem. The
study behind this talk aims to gain some insights about how children
learn to extract words from fluent speech.  However, the focus of this
talk will be describing a particular computational strategy that can
be useful for any other purpose.
}

\section{Introduction}
\includeslidex{s-intro}
\includeslidex{s-wreck-a-nice-beach}
\includeslidex{s-cues-intro}
\includeslidex{s-safran-etal}
\includeslidex{s-intro-segmentation}
\section{Predictability}
\includeslidex{s-sv-intro}
\includeslidex{s-measures-formula}
\includeslidex{s-measures-rev-and-size}
\includeslidex{s-sliding-window}
\includeslidex{s-tp-graph}
\includeslidex{s-mi-graph}
\includeslidex{s-sv-graph}
\includeslidex{s-h-graph}
\includeslidex{s-peak}
\includeslidex{s-combination}
\section{Evaluation}
\includeslidex{s-evaluation}
\includeslidex{s-eval-boundary-word-lexical}
\section{Recap}
\includeslidex{s-recap}
\includeslidex{s-algorithm}
\section{Results}
\includeslidex{s-results}
\includeslidex{s-example-solution}
\section{Summary}
\includeslidex{s-conclusions}

\appendix
\section{Appendix}
\includeslidexx{s-appendix-title}
\includeslidex{s-computational-modeling}
\includeslidex{s-computational-modeling2}
%\includeslidex{s-example-colors}
\includeslidex{s-measures-backwards}
\includeslidex{s-segmentation-difficult}

\end{document}
