\begin{frame}
\frametitle{Predictability}

\textcolor{blue}{\emph{\small
Predictability within units is high, predictability between units is low.}}
\vspace{1mm}

Given a sequence \xL{}\xR{}, where \xL{} and \xR{} are sequences of
phonemes:
\begin{itemize}
\item If \xL{} help us predict \xR, \xL\xR{} is likely to be part of a word.
\item If observing \xR{} after \xL{} is surprising it is likley that
there is a boundary between \xL{} and \xR{}.
\end{itemize}

\vspace{2mm}
\uncover<2->{An old method based on predictability, called
\emph{successor variety} (SV), is suggested by Haris (1955): 

\vspace{1mm}\emph{\small The morpheme boundaries are at the locations
where there is a high variety of possible phonemes that follow the
initial segment.} }\vspace{2mm}

\begin{columns}[t]
\begin{column}{0.45\textwidth}
\uncover<3->{\textcolor{blue}{read-}}\uncover<4->{\{a,e,i,j,o,s,y,-,',\$\} (\textcolor{red}{10})\\
\uncover<4->{
read\\
reads\\
reading\\
readjusted\\
...\\
}}
\end{column}

\begin{column}{0.45\textwidth}
\uncover<3->{\textcolor{blue}{readi-}}\uncover<4->{\{e, l, n\} (\textcolor{red}{3})\\
\uncover<4->{
reading\\
readied\\
readily\\
}}
\end{column}
\end{columns}
\vspace{2mm}

\end{frame}
