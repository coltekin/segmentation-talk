\begin{frame}
\frametitle{Distributional regularities}

\textcolor{blue}{\emph{\small
Predictability within units is high, predictability between 
units is low.}}
\vspace{2mm}

\uncover<2->{An old method using distributional regularities is suggested by 
(Haris, 1955): \emph{\small The morpheme boundaries are at the locations where
there is a high variety of possible phonemes that follow the segment.}
}\vspace{2mm}

\begin{columns}[t]
\begin{column}{0.45\textwidth}
\uncover<2->{\textcolor{blue}{read-}}\uncover<3->{\{a,e,i,j,o,s,y,-,',\$\} (\textcolor{red}{10})\\
\uncover<3->{
read\\
reads\\
reading\\
reader\\
readjusted\\
...\\
}}
\end{column}

\begin{column}{0.45\textwidth}
\uncover<2->{\textcolor{blue}{readi-}}\uncover<3->{\{e, l, n\} (\textcolor{red}{3})\\
\uncover<3->{
reading\\
readied\\
readily\\
}}
\end{column}
\end{columns}
\vspace{2mm}

\uncover<4->{\small There are a number of ways to formalize the concept:
including conditional probability, entropy, mutual information...}

\end{frame}
