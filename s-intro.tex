\section{The Problem of Language Acquisition}
\begin{frame}
\frametitle{The puzzle to solve}

\begin{columns}[t]
\begin{column}{0.5\textwidth}

\resizebox{!}{0.45\textheight}{
\begin{minipage}{0.5\textwidth}
\texttt{\small
ljuuzuibutsjhiuljuuz\\     %kittythatsrightkitty
ljuuztbzjubhbjompwfljuuz\\ %kittysayitagainlovekitty
xibutuibu\\                %whatsthat
ljuuz\\                    %kitty
epzpvxbounpsfnjmlipofz\\   %doyouwantmoremilkhoney
ljuuzljuuzephhjf\\         %kittykittydoggie
opnjxibuepftbljuuztbz\\    %nomiwhatdoesakittysay
xibuepftbljuuztbz\\        %whatdoesakittysay
ephhjfeph\\                %doggiedog
ephhjf\\                   %doggie
opnjxibuepftuifephhjftbz\\ %nomiwhatdoesthedoggiesay
xibuepftuifephhjftbz\\     %whatdoesthedoggiesay
mjuumfcbczcjsejf\\         %littlebabybirdie
cbczcjsejf\\               %babybirdie
zpvepoumjlfuibupof\\       %youdontlikethatone
plbznpnnzublfuijtpvu\\     %okaymommytakethisout
dpx\\                      %cow
uifdpxtbztnppnpp\\         %thecowsaysmoomoo
xibuepftuifdpxtbzopnj\\    %whatdoesthecowsaynomi
}
\end{minipage}
}

\end{column}
\begin{column}{0.5\textwidth}
\only<2->{
Children need to:
\begin{itemize}
\item segment the input to linguistic units (words,
      morphemes etc.).
\item assign meanings to these units.
\item figure out which combinations of these units are acceptable in the
      language.
\item ...
\end{itemize}
}
\end{column}
\end{columns}
\end{frame}
