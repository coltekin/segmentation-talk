\begin{frame}
\frametitle{The puzzle to solve}

\begin{columns}[t]
\begin{column}{0.5\textwidth}

\resizebox{!}{0.45\textheight}{
\begin{minipage}{0.5\textwidth}
\texttt{\small
ljuuzuibutsjhiuljuuz\\     %kittythatsrightkitty
ljuuztbzjubhbjompwfljuuz\\ %kittysayitagainlovekitty
xibutuibu\\                %whatsthat
ljuuz\\                    %kitty
epzpvxbounpsfnjmlipofz\\   %doyouwantmoremilkhoney
ljuuzljuuzephhjf\\         %kittykittydoggie
opnjxibuepftbljuuztbz\\    %nomiwhatdoesakittysay
xibuepftbljuuztbz\\        %whatdoesakittysay
ephhjfeph\\                %doggiedog
ephhjf\\                   %doggie
opnjxibuepftuifephhjftbz\\ %nomiwhatdoesthedoggiesay
xibuepftuifephhjftbz\\     %whatdoesthedoggiesay
mjuumfcbczcjsejf\\         %littlebabybirdie
cbczcjsejf\\               %babybirdie
zpvepoumjlfuibupof\\       %youdontlikethatone
plbznpnnzublfuijtpvu\\     %okaymommytakethisout
dpx\\                      %cow
uifdpxtbztnppnpp\\         %thecowsaysmoomoo
xibuepftuifdpxtbzopnj\\    %whatdoesthecowsaynomi
}
\end{minipage}
}

\end{column}
\begin{column}{0.5\textwidth}
\only<2->{
\begin{itemize}
\item<2-> No clear acoustic markers
\item<3-> No lexical knowledge
\item<4-> Large acoustic variation
\item<4-> Noise
\item<4-> Even a comprehensive lexicon is enough
%      \texttt{/6go/}:\\ 
%        \begin{tabular}{p{18mm}lp{18mm}}\texttt{/6go/}\newline `ago'& or &\texttt{/6 go/}\newline `a go'\end{tabular}\\
%      \texttt{/Itsnoz/}:\\ 
%      \begin{tabular}{p{18mm}lp{18mm}}\texttt{/Its~noz/}\newline `its
%      nose' & or & \texttt{/It~snoz/}\newline `it snows'\end{tabular}
\end{itemize}
}
\end{column}
\end{columns}
\end{frame}

\mode<article>{
I'd like to start the talk with a puzzle. What you see in this slide
will serve as a demonstrative example of the input the children
receive. Each line contains an English utterance, taken from actual
speech directed to infants. The spaces are removed from the
utterances. Because, as I just mentioned, fluent speech does not have
reliable boundary markers. I guess you are not convinced that they are
transcribed utterances in English without spaces between words.
Because, each letter is transformed systematically to another one.
This hopefully helps us put ourselves into a baby's shoes.  We know
quite a few English words, which helps us discover them when we see or
hear an unsegmented utterance. But, children start with no lexical
knowledge. Throughout this talk this text will serve as an
approximation to the input children receive during language
acquisition.

Having bombarded with sequences like that, children need to ...  For
this talk we will be interested in the very first problem: how can one
able to discover words from an input like that?
}
