\begin{frame}
\frametitle{How do children segment?}
A number of cues are well supporeted by psycholinguistic studies
\begin{itemize}
\item Acoustic cues
\begin{itemize}
    \item Lexical stress (Jusczyk et al., 1999)
    \item Pauses (Wightman et al., 1992)
    \item Allophonic alternations (Church, 1987).
    \item Coarticulation (Johnson \& Jusczyk, 2001).
    \item Vowel/consonant harmony (Suomi et al., 1997; van Kampen et al., 2008).
\end{itemize}
\item Words in isolation (Brent \& Siskind, 2001)
\item Phonotactics (Jusczyk et al., 1993)
\item Statistical regularities (Saffran et al., 1996)
\end{itemize}
\visible<2->{
\begin{beamercolorbox}[sep=0.5em,rounded=true]{whitebox}
All these cues are useful, but none of them is enough by itself.
\end{beamercolorbox}
}
\end{frame}
