\begin{frame}
\frametitle{Measures of (un)predictability}
%As an indication of a boundary betwen an initial sequence \xL, the
%sequence \xR, as well as successor variety we can use,
\begin{columns}[t]
\begin{column}{0.5\textwidth}
\begin{itemize}
\item Transitional probability
  \begin{equation*}
    \text{TP}(\xL,\xR) 
%                   = P(\xR|\xL) 
                   = \frac{P(\xL\xR)}{P(\xL)} 
%             \approx \frac{frequency(\xL\xR)}{frequency(\xL)}
  \end{equation*}
\item Pointwise mutual information
\begin{equation*}
    \textrm{MI}(\xL,\xR) = log_2 \frac{P(\xL\xR)}{P(\xL) P(\xR)}
\end{equation*}
\end{itemize}
\end{column}
\begin{column}{0.5\textwidth}
\begin{itemize}
\item Successor value
\begin{equation*}
\textrm{SV}(\xL) = \sum_{\xR\in A} c(\xL,\xR)
\end{equation*}
%where,
%\begin{equation*}
%c(\xL,\xR)=\begin{cases}%
%1 & \text{if substring \xL\xR occurs in the corpus}\\
%0 & \text{otherwise}\\
%\end{cases}
%\end{equation*}
\item Boundary entropy
\begin{equation*}\hspace{-10mm}
    \textrm{H}(\xL) = -\sum_{\xR\in A}
          P(\xR|\xL) \; log_2 P\left(\xR|\xL\right)
  \end{equation*}
\end{itemize}
\end{column}
\end{columns}

\vspace{2mm}\begin{beamercolorbox}[sep=0.2em,rounded=true]{whitebox}
\centering All these measures are related, but they are not the same.
\end{beamercolorbox}
\vspace{2mm}\begin{beamercolorbox}[sep=0.2em,rounded=true]{whitebox}
\centering This list is not exclusive, there are other measures of
(un)predictability.
\end{beamercolorbox}

\end{frame}
