\begin{frame}
\frametitle{Statistical regularities for segmentation}


\begin{center}
\end{center}
\begin{tikzpicture}[remember picture,overlay]
\node[yshift=-2cm] at (current page.north) {%
    \begin{minipage}{\textwidth}%
    Children very early in life (8-months) seem to be sensitive to
    statistical regularities between syllables (Saffran, Aslin,
    Newport 1996)
    \end{minipage}};
\only<2>{
\node[yshift=1.5cm] at (current page.center) 
     {Training: \texttt{bidakupadotigolabubidakugolabupadoti\ldots}};
}
\only<3->{
\node[yshift=1.5cm] at (current page.center) 
     {Training: \texttt{\bl{bidaku}\rd{padoti}\grn{golabu}\bl{bidaku}\grn{golabu}\rd{padoti}\ldots}};
\visible<4->{
\node[yshift=1cm,xshift=-2.5cm,anchor=east] at (current page.center) 
    {$TP(\bl{bi},\bl{da}) = 1$};
\node[yshift=1cm,xshift=3cm,anchor=west] at (current page.center) 
    {$TP(\grn{bu},\rd{pa}) = \frac{1}{3}$};
\draw[dotted]  ($(current page.center) + (2.53,1.3)$)
    rectangle
               ($(current page.center) + (3.32,1.72)$);
\draw[dotted]  ($(current page.center) + (-2.36,1.3)$)
    rectangle
               ($(current page.center) + (-3.15,1.72)$);
}
}
\visible<5->{
\draw[->,thick] ($(current page.center) + (0, 1)$) -- 
                node[above,sloped] {test G1: words}
                ($(current page.center) + (-3.5, -1)$);
\draw[->,thick] ($(current page.center) + (0, 1)$) -- 
                node[above,sloped] {test G2: non-words}
                ($(current page.center) + (3.5, -1)$);
\node[yshift=-1.3cm,xshift=-5mm,anchor=east] at (current page.center) 
    {\texttt{\rd{padoti}\bl{bidaku}\grn{golabu}\rd{padoti} \ldots}};
\node[yshift=-1.3cm,xshift=5mm,anchor=west] at (current page.center) 
    {\texttt{pagolabidotikugobdalaubu \ldots}};
}
\visible<6->{
\node[yshift=1.5cm] at (current page.south) {%
%    \begin{minipage}{\textwidth}%
\begin{beamercolorbox}[sep=0.5em,rounded=true]{whitebox}
    Children showed preference towards the `words' that are used in
    the training phase.
    \end{beamercolorbox}};
%    \end{minipage}};
}
\end{tikzpicture}

\end{frame}

%\begin{itemize}
%\item Infants are habituated to artificial speech sequences built 
%    from a simple vocabulary, such as `\emph{bidaku}', `\emph{padoti}',
%    `\emph{golabu}'. Resulting in sequences like:
%          `,
%\item The syllable transition probabilities within the words were 1.
%      That is, wthin the words the next syllable was deterministic. 
%\item The syllable transition probabilities between the words were 1/3. 
%\item After habituation, children are tested with either stequences of `words', 
%    or random sequences formed by the same syllables.
%\item On the basis of very short training 8-month-old infants attended
%      familiar examples significantly longer than the unfamiliar ones.  
%\end{itemize}
